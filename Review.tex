\documentclass[titlepage,onecolumn,showpacs,nofootinbib,aps,superscriptaddress,eqsecnum,prd,notitlepage,showkeys,12pt]{article}
\usepackage[utf8]{inputenc}
\usepackage[T1]{fontenc}
\usepackage{amssymb}
\usepackage{adjustbox}
\usepackage{amsmath}
\usepackage{textgreek}
\usepackage[outdir=./]{epstopdf}
\usepackage{graphicx}
\usepackage{dcolumn}
\usepackage{mathtools}
\usepackage{color}
\usepackage{hyperref}
\usepackage{pdflscape}
\usepackage{authblk}	
\usepackage{booktabs}
\usepackage{float}
\usepackage{fancyvrb}
\usepackage{natbib}
\usepackage{amsmath}
\usepackage{pdflscape}
\usepackage[utf8]{inputenc}
\newcounter{defcounter}
\setcounter{defcounter}{0}
\lefthyphenmin4
\righthyphenmin4
\usepackage{geometry}

\geometry{
	a4paper,
	total={210mm,297mm},
	left=25.4mm,
	right=25.4mm,
	top=25.4mm,
	bottom=25.4mm,
}
\usepackage{amsthm}
%\title{A critical review of behaviour change frameworks}
%\title{A critical review of behaviour change frameworks}
\author[1,3]{Krishane Patel}
\author[2,3]{Magda Osman}
\author[1]{Susan Michie}
\affil[1]{Clinical, Educational and Health Psycholgy, University College London}
\affil[2]{Queen Mary University, London, UK}
\affil[3]{Food Standards Agency, London, UK}
\date{}
\begin{document}

\flushbottom
\maketitle
% * <john.hammersley@gmail.com> 2015-02-09T12:07:31.197Z:
%
%  Click the title above to edit the author information and abstract
%
\section{Introduction}
The purpose of this project is to critically review behaviour change frameworks in their use and function. Policy-makers are ever increasingly using behaviour change in public policy creation, ever since the creation of the Behavioural Insights Team or `\textit{nudge unit}' in the UK government \citep{MINDSPACE}. With more nudge units being implemented and embedded in govermental departments across the world \citep{shafir2013behavioral}.\\
\indent Policy-makers are interested in how to implement behaviour change given a specific behavioural problem, specifically they seek to maximise the effectiveness of behaviour change techniques (BCT's), or more specifically, minimize the delta between the desired and observed outcome as executed through policy through the use of BCT's. BCT's are typically used in public policy without any reliable format of how and when to utilise and implement, utilising commonsense models of behaviour to guide the design process \citep{michie2009specifying}.\\
\indent In this demand space, behavioural change frameworks (BCF's) were designed and created specifically for the function of implementation within public policy (for example, MINDSPACE \citep{MINDSPACE} or the Behaviour Change Wheel \citep{BCW}). BCF's are based upon solid theories of behaviour \citep{adom2018theoretical} and can be considered as structural supports that act as `\textit{blueprints}' or a guide for research \citep{grant2014understanding}.
This process is governed by BCF's, but there exists a plethora of BCF's where each design provides given affordances and functions.
\subsection{Rationale}
 Intervention designers and policy-makers often do not use existing frameworks as a basis for developing new policies or interventions, or even to analyse and explain why certain policies/interventions have failed while others succeed \citep{BCW}. One reason may be due to a delta between the needs of policy-makers and interventionists and the BCF's functions and affordances. Where sub-optimal BCF's provide only a partial view of the policy-design process, either at a process level, to optimise the policy; or at the diagnostic level, in understanding the behavioural processes and restricting the BCT set \citep{davis2015theories}. A secondary reason may be due to comprehensiveness of the frameworks themselves, BCF's should offer a comprehensive view of behavioural processes that could be applied to any and all (potential) interventions or policies \citep{BCW}. BCF's should optimally seek to provide an exhaustive, clear understanding of the underlying field of inquiry. A third potential factors is a framework's coherence, that is to say it's constituent categories must occupy the same specifity levels, i.e. some categories may not be overly broad whilst others specific, and should be discussed from at the highest ordinal position.\\
\indent Given the ever growing usage of behavioural science in policy, we seek to critically review BCF's to optimise the use of BCT's and behaviour change science for  greater efficiency in public policy.
\subsection{Objectives}
This paper seeks to critically review the BCF space to establish how far each meets the criteria of usefulness, and to identify a comprehensive list of intervention descriptors at a level of generality that is usable by intervention designers and policy makers. Here it is necessary to distinguish the areas of usefulness.
\begin{itemize}
    \item Comprehensiveness - A BCF should be comprehensive in its explanation and description of the area of inquiry it represents.
    \item Coherence - A BCF should be coherent and provide representations at the highest ordinal position 
    \item 
\end{itemize}
\section{Methods}
\subsection{Eligibility criteria}
\subsection{Information sources}
\subsection{Search Strategy}
James Thomas search model (see XXXX)
Same as BCW implementation. 
\newpage
\bibliographystyle{apa}
\bibliography{Review.bib}
\end{document}